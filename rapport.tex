\documentclass{article}

% Language setting
\usepackage[french]{babel}

% Set page size and margins
\usepackage[letterpaper,top=2cm,bottom=2cm,left=3cm,right=3cm,marginparwidth=1.75cm]{geometry}

% Useful packages
\usepackage{amsmath}
\usepackage{graphicx}
\usepackage[colorlinks=true, allcolors=blue]{hyperref}

\title{POOIG : ProtectiveTowers}
\author{Mehdi Chater \& Mathéo Piget}
\date{16 janvier 2024}

\begin{document}
\maketitle


\section{Introduction}

Blabla d'introduction

\href{https://www.overleaf.com/learn}{lien hypertext}

\section{Grand Titre}

\subsection{Petit titre du grand titre}

blabla

\subsection{Petit titre du grand titre 2}

blabla

\begin{figure}
\centering
\includegraphics[width=0.25\linewidth]{fichier.jpg} % mettre fichier image
\caption{\label{fig:frog}Mettre nom figure}
\end{figure}

\begin{table}
\centering
\begin{tabular}{l|r}
colonne1 & colonne2 \\\hline
truc1 & 85 \\
truc2 & 64
\end{tabular}
\caption{\label{tab:widgets}Faire un tableau}
\end{table}

\subsection{encore un titre}

blabla 

\subsection{How to add Lists}

You can make lists with automatic numbering \dots

\begin{enumerate}
\item Like this,
\item and like this.
\end{enumerate}
\dots or bullet points \dots
\begin{itemize}
\item Like this,
\item and like this.
\end{itemize}


\bibliographystyle{alpha}
\bibliography{sample}

\end{document}